\documentclass[letterpaper, 11pt]{aq-hw}

\usepackage{lipsum}

\author{Student name}
\course{Class name}
\hwtype{Problem Set}
\hwnumber{7}

\begin{document}

\paragraph{1.} \lipsum[1]

\paragraph{2.} \lipsum[2]

\begin{figure}[ht]
	\centering
	\begin{tikzpicture}
		\draw(0, 0) -- (0, -2)
		node[pos=0] {\textbullet};
		\draw (0, -2) -| (5, 2)
		node[pos=0, below] {$(0, c)$}
		node[pos=0.5, below] {$(b, c)$}
		node[pos=1, above] {$(b, d)$}
		;
		\draw (5, 2) -| (0, 0)
		node[pos=0.5, above] {$(0, d)$}
		;
		\draw(0, 0.5) -| (1, -0.5)
		node[pos=0.25, above] {$\varepsilon$}
		node[pos=0.75, right] {$\varepsilon$}
		node[pos=0.75, left] {$Q_\varepsilon$}
		;
		\draw(1, -0.5) -- (0, -0.5);
		\draw[dashed](1, 2) |- (5, 0.5);
		\draw[dashed](1, -2) |- (5, -0.5);
	\end{tikzpicture}
	\hfill
	\begin{tikzpicture}
		\draw(0, 0) -| (4, 3)
		node[pos=0] {\textbullet}
		node[pos=0.5, below] {$(b, 0)$}
		;
		\draw (4, 3) -| (0, 0)
		node[pos=0, above] {$(b, d)$}
		node[pos=0.5, above] {$(0, d)$}
		;
		\draw (0, 1) -| (1, 0)
		node[pos=0.25, above] {$\varepsilon$}
		node[pos=0.75, right] {$\varepsilon$}
		node[pos=0.75, left] {$Q_\varepsilon$}
		;
		\draw[dashed](1, 3) |- (4, 1);
	\end{tikzpicture}
	\label{fig:epsilon-rect}
\end{figure}

\paragraph{3.} \lipsum[3]

\paragraph{4.} \lipsum[4]

\end{document}