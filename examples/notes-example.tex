\documentclass[11pt, letterpaper]{aq-notes}

\title{\scshape The Fourier transform}
\author{Arthur Lei Qiu}
\date{7 November 2019}
\course{MAT495}

\newcommand{\schwartz}{\mathcal{S}}

\begin{document}

\maketitle
%\tableofcontents

\section{The Fourier transform}

\subsection{Introduction}

Roughly speaking, the Fourier transform is a ``continuous'' version of the Fourier series. Recall that the $n^\text{th}$ Fourier coefficient of an integrable function $f \colon [0, 1] \to \C$ is given by
\[
	a_n = \int_0^1 f(x)e^{-2\pi inx}\,dx.
\]
The Fourier transform can be thought of as indexing Fourier coefficients over $\R$ instead of $\Z$ (that is, allowing $a_\xi$ for $\xi \in \R$).

\begin{definition}
	The \textbf{Fourier transform} of a function $f \colon \R \to \C$ is the function $\hat{f} \colon \R \to \C$ defined by
	\begin{align}\label{eq:ftransform}
		\hat{f}(\xi) = \int_{-\infty}^\infty f(x)e^{-2\pi i\xi x}\,dx,
	\end{align}
	the integral being taken in the Riemann sense.
\end{definition}
Of course, this definition is purely formal; it is not clear if the integral even converges. To make sense of this, we need some hypotheses on the smoothness and decay of $f$.

\subsection{Schwartz functions}

\begin{definition}
	A smooth function $f \colon \R \to \C$ is called a \textbf{Schwartz function} if $f$ and all of its derivatives are rapidly decreasing, in the sense that
	\begin{align}\label{eq:rapid-dec}
		\sup_{x \in \R} |x^kf^{(\ell)}(x)| < \infty, \qquad \forall k, \ell \geq 0.
	\end{align}
	The collection of all such functions is called the \textbf{Schwartz space} on $\R$ and denoted by $\schwartz$.
\end{definition}

\begin{proposition}\label{prop:schwartz-prop}
	The following are true about Schwartz space.
	\begin{enumerate}[label=(\roman*)]
		\item $\schwartz$ forms a vector space over $\C$. Moreover, $\schwartz$ is closed under differentiation and multiplication by polynomials.
		\item $C^\infty_c(\R, \C) \subsetneq \schwartz \subsetneq C^\infty(\R, \C)$ and $\schwartz \subsetneq \mathcal{M}$.
	\end{enumerate}
\end{proposition}

\begin{proof}
	\textbf{(i)} This is mostly an application of the triangle inequality. Fix $k, \ell \geq 0$. Given $f, g \in \schwartz$ and $a \in \C$, we have
	\[
		\sup_{x \in \R} \left|x^k(af + g)^{(\ell)}(x)\right| \leq \sup_{x \in \R} \left(\left|x^k(af)^{(\ell)}(x)\right| + \left|x^kg^{(\ell)}(x)\right|\right) \leq |a|\sup_{x \in \R} |x^kf^{(\ell)}(x)| + \sup_{x \in \R} |x^kg^{(\ell)}(x)|.
	\]
	Since $|x^k(f')^{(\ell)}(x)| = |x^kf^{(\ell + 1)}(x)|$, Schwartz space is closed under differentiation. To show that $\schwartz$ is closed under multiplication by polynomials, we show that for all $f \in \schwartz$, we have $xf(x) \in \schwartz$, and using this and the fact that $\schwartz$ is a vector space, we obtain the result. We can show inductively that
	\[
		(xf(x))^{(\ell)} = \ell f^{(\ell - 1)}(x) + xf^{(\ell)}(x), \qquad \forall \ell \geq 0,
	\]
	and hence,
	\[
		\sup_{x \in \R} |x^k(xf(x))^{(\ell)}| \leq \ell\sup_{x \in \R}|x^kf^{(\ell - 1)}(x)| + \sup_{x \in \R}|x^{k + 1}f^{(\ell)}(x)| < \infty.
	\]

	\textbf{(ii)} An example of a smooth function that is not Schwartz is given by $e^x$. The inclusion $C^\infty_c(\R, \C) \subseteq \schwartz$ is a consequence of the extreme value theorem, and an example of a Schwartz function with non-compact support is the Gaussian
	\[
		f(x) = e^{-ax^2}, \qquad a > 0.
	\]
	To see this, observe that every derivative of $f$ is of the form $f^{(\ell)}(x) = P_\ell(x)f(x)$ for some polynomial $P_\ell$, and that
	\[
		\sup_{x \in \R} |x^kf(x)| < \infty, \qquad \forall k \geq 0.
	\]
	Finally, if $f(x)$ is bounded by $\alpha$ and $x^2f(x)$ is bounded by $\beta$, then
	\[
		(1 + x^2)|f(x)| \leq \alpha + \beta,
	\]
	hence $\schwartz \subseteq \mathcal{M}$. On the other hand, the function $e^{-|x|}$ is moderately decreasing (in fact, rapidly decreasing) but not Schwartz because it is not differentiable at $0$.
\end{proof}

Since Schwartz functions are moderately decreasing, their Fourier transform is well-defined, and it turns out that Schwartz functions play particularly nicely with the Fourier transform. In the following statements, the notation $f(x) \mapsto \hat{f}(\xi)$ means that $\hat{f}$ is the Fourier transform of $f$.

\begin{proposition}\label{prop:fourier-prop}
	Let $f \in \schwartz$. The following are true:
	\begin{enumerate}[label=(\roman*)]
		\item If $h \in \R$, then
		\[
			f(x + h) \mapsto \hat{f}(\xi)e^{2\pi i\xi h}
		\]
		and
		\[
			f(x)e^{-2\pi i hx} \mapsto \hat{f}(\xi + h).
		\]
		\item If $\delta > 0$, then
		\[
			f(\delta x) \mapsto \f{1}{\delta}\hat{f}\left(\f{\xi}{\delta}\right).
		\]
		\item We have
		\[
			f'(x) \mapsto 2\pi i \xi\hat{f}(\xi).
		\]
		Furthermore, $\hat{f}$ is differentiable, and
		\[
			-2\pi i xf(x) \mapsto (\hat{f})'(\xi).
		\]
	\end{enumerate}
\end{proposition}

\begin{proof}
	Exercise.
\end{proof}

\end{document}
